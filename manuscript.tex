%%%%%%%%%%%%%%%%%%%%%%%%%%%%%%%%%%%%%%%%%%%%%%%%%%%%%%%%%%%%%%%%%%%%%
%% This is a (brief) model paper using the achemso class
%% The document class accepts keyval options, which should include
%% the target journal and optionally the manuscript type. 
%%%%%%%%%%%%%%%%%%%%%%%%%%%%%%%%%%%%%%%%%%%%%%%%%%%%%%%%%%%%%%%%%%%%%
\documentclass[journal=jacsat,manuscript=article]{achemso}

%%%%%%%%%%%%%%%%%%%%%%%%%%%%%%%%%%%%%%%%%%%%%%%%%%%%%%%%%%%%%%%%%%%%%
%% Place any additional packages needed here.  Only include packages
%% which are essential, to avoid problems later. Do NOT use any
%% packages which require e-TeX (for example etoolbox): the e-TeX
%% extensions are not currently available on the ACS conversion
%% servers.
%%%%%%%%%%%%%%%%%%%%%%%%%%%%%%%%%%%%%%%%%%%%%%%%%%%%%%%%%%%%%%%%%%%%%
\usepackage[version=3]{mhchem} % Formula subscripts using \ce{}
\usepackage{algorithm}
\usepackage{algpseudocode}
%%%%%%%%%%%%%%%%%%%%%%%%%%%%%%%%%%%%%%%%%%%%%%%%%%%%%%%%%%%%%%%%%%%%%
%% If issues arise when submitting your manuscript, you may want to
%% un-comment the next line.  This provides information on the
%% version of every file you have used.
%%%%%%%%%%%%%%%%%%%%%%%%%%%%%%%%%%%%%%%%%%%%%%%%%%%%%%%%%%%%%%%%%%%%%
%%\listfiles

%%%%%%%%%%%%%%%%%%%%%%%%%%%%%%%%%%%%%%%%%%%%%%%%%%%%%%%%%%%%%%%%%%%%%
%% Place any additional macros here.  Please use \newcommand* where
%% possible, and avoid layout-changing macros (which are not used
%% when typesetting).
%%%%%%%%%%%%%%%%%%%%%%%%%%%%%%%%%%%%%%%%%%%%%%%%%%%%%%%%%%%%%%%%%%%%%
\newcommand*\mycommand[1]{\texttt{\emph{#1}}}

%%%%%%%%%%%%%%%%%%%%%%%%%%%%%%%%%%%%%%%%%%%%%%%%%%%%%%%%%%%%%%%%%%%%%
%% Meta-data block
%% ---------------
%% Each author should be given as a separate \author command.
%%
%% Corresponding authors should have an e-mail given after the author
%% name as an \email command. Phone and fax numbers can be given
%% using \phone and \fax, respectively; this information is optional.
%%
%% The affiliation of authors is given after the authors; each
%% \affiliation command applies to all preceding authors not already
%% assigned an affiliation.
%%
%% The affiliation takes an option argument for the short name.  This
%% will typically be something like "University of Somewhere".
%%
%% The \altaffiliation macro should be used for new address, etc.
%% On the other hand, \alsoaffiliation is used on a per author basis
%% when authors are associated with multiple institutions.
%%%%%%%%%%%%%%%%%%%%%%%%%%%%%%%%%%%%%%%%%%%%%%%%%%%%%%%%%%%%%%%%%%%%%
\author{Nicholas Runchie}
\altaffiliation{EaSTCHEM School of Chemistry, University of Edinburgh, EH9 3FJ}
\author{Antonia S.J.S. Mey}
\altaffiliation{EaSTCHEM School of Chemistry, University of Edinburgh, EH9 3FJ}
\email{antonia.mey@ed.ac.uk}
\phone{xxx}


%%%%%%%%%%%%%%%%%%%%%%%%%%%%%%%%%%%%%%%%%%%%%%%%%%%%%%%%%%%%%%%%%%%%%
%% The document title should be given as usual. Some journals require
%% a running title from the author: this should be supplied as an
%% optional argument to \title.
%%%%%%%%%%%%%%%%%%%%%%%%%%%%%%%%%%%%%%%%%%%%%%%%%%%%%%%%%%%%%%%%%%%%%
\title[SILVR: Molecular Generation for binding modes]
  {SILVR: Conditional diffusion model for molecule generation without additional training}

%%%%%%%%%%%%%%%%%%%%%%%%%%%%%%%%%%%%%%%%%%%%%%%%%%%%%%%%%%%%%%%%%%%%%
%% Some journals require a list of abbreviations or keywords to be
%% supplied. These should be set up here, and will be printed after
%% the title and author information, if needed.
%%%%%%%%%%%%%%%%%%%%%%%%%%%%%%%%%%%%%%%%%%%%%%%%%%%%%%%%%%%%%%%%%%%%%
\abbreviations{}
\keywords{ML, generative models, docking}

%%%%%%%%%%%%%%%%%%%%%%%%%%%%%%%%%%%%%%%%%%%%%%%%%%%%%%%%%%%%%%%%%%%%%
%% The manuscript does not need to include \maketitle, which is
%% executed automatically.
%%%%%%%%%%%%%%%%%%%%%%%%%%%%%%%%%%%%%%%%%%%%%%%%%%%%%%%%%%%%%%%%%%%%%
\begin{document}

%%%%%%%%%%%%%%%%%%%%%%%%%%%%%%%%%%%%%%%%%%%%%%%%%%%%%%%%%%%%%%%%%%%%%
%% The "tocentry" environment can be used to create an entry for the
%% graphical table of contents. It is given here as some journals
%% require that it is printed as part of the abstract page. It will
%% be automatically moved as appropriate.
%%%%%%%%%%%%%%%%%%%%%%%%%%%%%%%%%%%%%%%%%%%%%%%%%%%%%%%%%%%%%%%%%%%%%
\begin{tocentry}

Some journals require a graphical entry for the Table of Contents.
This should be laid out ``print ready'' so that the sizing of the
text is correct.

Inside the \texttt{tocentry} environment, the font used is Helvetica
8\,pt, as required by \emph{Journal of the American Chemical
Society}.

The surrounding frame is 9\,cm by 3.5\,cm, which is the maximum
permitted for  \emph{Journal of the American Chemical Society}
graphical table of content entries. The box will not resize if the
content is too big: instead it will overflow the edge of the box.

This box and the associated title will always be printed on a
separate page at the end of the document.

\end{tocentry}

%%%%%%%%%%%%%%%%%%%%%%%%%%%%%%%%%%%%%%%%%%%%%%%%%%%%%%%%%%%%%%%%%%%%%
%% The abstract environment will automatically gobble the contents
%% if an abstract is not used by the target journal.
%%%%%%%%%%%%%%%%%%%%%%%%%%%%%%%%%%%%%%%%%%%%%%%%%%%%%%%%%%%%%%%%%%%%%
\begin{abstract}
I'll be an abstract when I grow up. 
\end{abstract}

%%%%%%%%%%%%%%%%%%%%%%%%%%%%%%%%%%%%%%%%%%%%%%%%%%%%%%%%%%%%%%%%%%%%%
%% Start the main part of the manuscript here.
%%%%%%%%%%%%%%%%%%%%%%%%%%%%%%%%%%%%%%%%%%%%%%%%%%%%%%%%%%%%%%%%%%%%%
\section{Introduction}
- Equivariant diffusion models have been very promising in predicting 3D structures of drug-like molecules. 
- This does not make them suitable candidates for a target binding site. 
- We combine equivariant diffusion models~\cite{huang2022mdm} with Iterative latent variable refinement proposed in Denoising Diffusion probabilistic models~\cite{choi2021ilvr}.
- This allows the generation of new molecules in the shape of the binding site using information from existing fragments.

\section{Theory}
\subsectin{Unconditional Equivariant Denoising Diffusion probabilistic models}
The main idea behind diffusion models is for a neural netowrk to learn the reverse of a diffusion process, often referred to as denoising. If you start out with a set of data $\mathbf{x}$ and add noise to this data over a time interval t = [0,\ldots,T] you can define the following distribution:
\begin{equation}
  p(\mathbf{z}_t)|\mathbf{x} = \mathcal{N}(\mathbf{z}_t|\alpha_t,\amthbf{x}_t,\sigma^2_t\mathbf{I} ) 
\end{equation}

We do not need to know $\matbf{x}$ and want to approximate this using a neural network $\phi$, according to $\mathbf{x} = \phi(\mathbf{z}_t,t)$
\subsection{Iterative Latent Variable Refinement as a conditioning tool}
Choi et al.~\cite{choi2021ilvr} introduced a way to condition DDPMs by sampling their images from the conditional distribution $p(x_0|c)$, with the condition c:
\begin{equation}
    P_\theta(x_0|c) = \int p_\theta(x_{0:T}|c)Dx_{1:T}
\end{equation}
\subsection{Selective Iterative Variable Refinement}

\begin{algorithm}
\caption{SILVR}\label{alg:cap}
\begin{algorithmic}
\State Sample $x_T \sim \mathcal{N}(0,I)$
\For {$t$ = $T,...,1$}
    \State $z \sim \mathcal{N}(0,I)$
    \State $x_{t-1}^\prime \sim p_\theta(x_{t-1}^\prime|x_t)$
    \State $y_{t-1} \sim q(y_{t-1}^\prime|y)$
    \State $x_{t-1} \gets x_{t-1}^\prime - k x_{t-1}^\prime +k y_{t-1}$ \Comment{SILVR equation}

\EndFor
\State Sample $x,h \sim p(x,h|z_0)$ 
\end{algorithmic}
\end{algorithm}

\subsecsection{Algorithm}
Both of these are wrong. I have just quickly tried to transcribe them and try and modify them. I think the ILVR style is easier to read and convey the point clearly. 
\begin{algorithm}
\caption{Sampling from EDM (coppied from EDM)}\label{alg:cap}
\begin{algorithmic}
\State Sample $z_T \sim \mathcal{N}(0,I)$
\For {$t$ in $T, T-1,...,1$ where $s=t-1$}
    \State Sample $\epsilon \sim \mathcal{N}(0,I)$
    \State Subtract center of gravity from $\epsilon^{(x)}$ in $\epsilon = [\epsilon^{(x)}, \epsilon^{(h)}]$
    \State $z_s^\prime = \frac{1}{\alpha_{t|s}}z_t - \frac{\sigma^2_{t|s}}{\alpha_{t|s}\sigma_t} \cdot \phi(z_t,t) + \sigma_{t \to s} \cdot \epsilon$
    \State $z_s \gets z_s^\prime - \lambda z_s^\prime +\lambda z_{ref}$ \Comment{SILVR equation}
\EndFor
\State Sample $x,h \sim p(x,h|z_0)$ 
\end{algorithmic}
\end{algorithm}





\section{Methods}

\section{Results}


\section{Discussion and Conclusions}



%%%%%%%%%%%%%%%%%%%%%%%%%%%%%%%%%%%%%%%%%%%%%%%%%%%%%%%%%%%%%%%%%%%%%
%% The "Acknowledgement" section can be given in all manuscript
%% classes.  This should be given within the "acknowledgement"
%% environment, which will make the correct section or running title.
%%%%%%%%%%%%%%%%%%%%%%%%%%%%%%%%%%%%%%%%%%%%%%%%%%%%%%%%%%%%%%%%%%%%%
\begin{acknowledgement}
The authors thank \ldots. 

\end{acknowledgement}



%%%%%%%%%%%%%%%%%%%%%%%%%%%%%%%%%%%%%%%%%%%%%%%%%%%%%%%%%%%%%%%%%%%%%
%% The appropriate \bibliography command should be placed here.
%% Notice that the class file automatically sets \bibliographystyle
%% and also names the section correctly.
%%%%%%%%%%%%%%%%%%%%%%%%%%%%%%%%%%%%%%%%%%%%%%%%%%%%%%%%%%%%%%%%%%%%%
\bibliography{achemso-demo}

\end{document}